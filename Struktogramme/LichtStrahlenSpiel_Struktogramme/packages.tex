\usepackage[T1]{fontenc}
\usepackage[ngerman]{babel}
\usepackage{graphicx}       %wegen resizebox

\usepackage{fouriernc}      %rm-Schriftart: Fourier New Century
\usepackage{avant}          %sf-Schriftart: Avant Garde
\usepackage{courier}        %tt-Schriftart: Courier
\usepackage{struktex}       %Struktogramme
\usepackage{amsmath}        %Mathe-Funktionen

\usepackage{microtype}      %Font-Rendering verbessern

\renewcommand{\pTrue}{J}
\renewcommand{\pFalse}{N}

\usepackage[a4paper,margin=1.5cm,noheadfoot]{geometry} %Seitenr"ander

\usepackage{relsize}        %relative Schriftgr"o"se

\usepackage{xcolor}         %Farbe bekennen
\definecolor{JunghansBlau}{cmyk}{1,0.7,0,0}

\usepackage[
    bookmarks,
    bookmarksnumbered,    
    bookmarksopen=true,
    bookmarksopenlevel=1,
    colorlinks=true,
    linkcolor=JunghansBlau, % einfache interne Verkn"upfungen
    anchorcolor=JunghansBlau,% Ankertext
    citecolor=JunghansBlau, % Verweise auf Literaturverzeichniseinträge im Text
    filecolor=JunghansBlau, % Verkn"upfungen, die lokale Dateien "offnen
    menucolor=JunghansBlau, % Acrobat-Men"upunkte
    urlcolor=JunghansBlau, 
    pdftex,
    plainpages=false, % zur korrekten Erstellung der Bookmarks
    pdfpagelabels=true, % zur korrekten Erstellung der Bookmarks
    hypertexnames=false, % zur korrekten Erstellung der Bookmarks
    linktocpage % Seitenzahlen anstatt Text im Inhaltsverzeichnis verlinken
]{hyperref}
% Befehle, die Umlaute ausgeben, f"uhren zu Fehlern, wenn sie hyperref als Optionen "ubergeben werden
\hypersetup{
    pdftitle={\Title},
    pdfauthor={\Author},
    pdfcreator={\Author},
    pdfsubject={\Title},
    pdfkeywords={\Title},
}